\documentclass{article}

\usepackage{amssymb, amsmath, multicol}
\usepackage{graphicx}
\usepackage{float}
\usepackage{wrapfig}
\usepackage[utf8]{inputenc}
\usepackage[T1,T2A]{fontenc}
\usepackage[russian]{babel}
\begin{document}
Дано: $\frac{sin(x)}{cos(x)}$\bigskip Если Земля плоская, то очевидно:

$(cos(x))'$ = $(-1 \cdot (sin(x))) \cdot 1$

\bigskip Ничего не понял, но очень интересно:

$(sin(x))'$ = $(cos(x)) \cdot 1$

\bigskip Любому советскому первокласснику очевидно, что

$(\frac{sin(x)}{cos(x)})'$ = $\frac{((cos(x)) \cdot 1) \cdot (cos(x)) - (sin(x)) \cdot ((-1 \cdot (sin(x))) \cdot 1)}{(cos(x)) \cdot (cos(x))}$

\bigskip После очевидных упрощений имеем:

$\frac{(cos(x)) \cdot (cos(x)) - (sin(x)) \cdot (-1 \cdot (sin(x)))}{(cos(x)) \cdot (cos(x))}$
\end{document}