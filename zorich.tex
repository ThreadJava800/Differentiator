\documentclass{article}

\usepackage{amssymb, amsmath, multicol}
\usepackage{graphicx}
\usepackage{float}
\usepackage{wrapfig}
\usepackage[utf8]{inputenc}
\usepackage[T1,T2A]{fontenc}
\usepackage[russian]{babel}
\begin{document}
Дано: ${cos(5 \cdot {x}^{3})}^{2} \cdot sin(3x)$\bigskip График функции ${cos(5 \cdot {x}^{3})}^{2} \cdot sin(3x)$имеет вид:

\begin{figure}[h]\center{\includegraphics[width=100mm]{graph.png}}\label{fig:t}\end{figure}\bigskip Взял с первой страницы в гугле:

$(3x)'$ = $0x + 3 \cdot 1$

\bigskip Любому советскому первокласснику очевидно, что

$(sin(3x))'$ = $cos(3x) \cdot (0x + 3 \cdot 1)$

\bigskip По теореме Дашкова-Гущина:

$({x}^{3})'$ = $1 \cdot 3 \cdot {x}^{2}$

\bigskip Заметим, что

$(5 \cdot {x}^{3})'$ = $0 \cdot {x}^{3} + 5 \cdot 1 \cdot 3 \cdot {x}^{2}$

\bigskip Любому советскому первокласснику очевидно, что

$(cos(5 \cdot {x}^{3}))'$ = $-1 \cdot sin(5 \cdot {x}^{3}) \cdot (0 \cdot {x}^{3} + 5 \cdot 1 \cdot 3 \cdot {x}^{2})$

\bigskip Я бы давно бы вас убил за это количество переменных А, ещё на стадии двух переменных А!

$({cos(5 \cdot {x}^{3})}^{2})'$ = $-1 \cdot sin(5 \cdot {x}^{3}) \cdot (0 \cdot {x}^{3} + 5 \cdot 1 \cdot 3 \cdot {x}^{2}) \cdot 2 \cdot {cos(5 \cdot {x}^{3})}^{1}$

\bigskip Я бы давно бы вас убил за это количество переменных А, ещё на стадии двух переменных А!

$({cos(5 \cdot {x}^{3})}^{2} \cdot sin(3x))'$ = $-1 \cdot sin(5 \cdot {x}^{3}) \cdot (0 \cdot {x}^{3} + 5 \cdot 1 \cdot 3 \cdot {x}^{2}) \cdot 2 \cdot {cos(5 \cdot {x}^{3})}^{1} \cdot sin(3x) + {cos(5 \cdot {x}^{3})}^{2} \cdot cos(3x) \cdot (0x + 3 \cdot 1)$

\bigskip После очевидных упрощений имеем:

$-1 \cdot sin(5 \cdot {x}^{3}) \cdot 5 \cdot 3 \cdot {x}^{2} \cdot 2 \cdot cos(5 \cdot {x}^{3}) \cdot sin(3x) + {cos(5 \cdot {x}^{3})}^{2} \cdot cos(3x) \cdot 3$
\end{document}