\documentclass{article}
\usepackage{amssymb, amsmath, multicol}
\usepackage[utf8]{inputenc}
\usepackage[T1,T2A]{fontenc}
\usepackage[russian]{babel}
\begin{document}
Совершенно очевидно, что\n\n
\\$(sin(x))'$ = $(cos(B)) \cdot A$\\\\ где:\\
A = $1.00$\\
B = $x$\\
C = $0.00$\\
Совершенно очевидно, что\n\n
\\$(cos(x))'$ = $(D \cdot C) \cdot (sin(A))$\\\\ где:\\
A = $x$\\
B = $0.00$\\
C = $1.00$\\
D = $-1.00$\\
Это преобразование позаимствуем из вступительных испытаний в советские ясли:\n\n
\\$(\frac{sin(x)}{cos(x)})'$ = $\frac{(((cos(B)) \cdot D) \cdot (cos(B))) - ((sin(B)) \cdot ((E \cdot D) \cdot (sin(B))))}{{cos(B)}^{A}}$\\\\ где:\\
A = $2.00$\\
B = $x$\\
C = $0.00$\\
D = $1.00$\\
E = $-1.00$\\
$\frac{(((cos(B)) \cdot D) \cdot (cos(B))) - ((sin(B)) \cdot ((E \cdot D) \cdot (sin(B))))}{{cos(B)}^{A}}$\\\\ где:\\
A = $2.00$\\
B = $x$\\
C = $0.00$\\
D = $1.00$\\
E = $-1.00$\\

\end{document}