\documentclass{article}

\usepackage{amssymb, amsmath, multicol}
\usepackage{graphicx}
\usepackage{float}
\usepackage{wrapfig}
\usepackage[utf8]{inputenc}
\usepackage[T1,T2A]{fontenc}
\usepackage[russian]{babel}
\begin{document}
Дано: $({B}^{2.00}) \cdot A$, где:

\bigskip\qquad A = $sin(3.00 \cdot ({x}^{5.00}))$

\qquad B = $cos(({x}^{6.00}) \cdot 3.00)$

\bigskip График функции $({B}^{2.00}) \cdot A$, где:

\bigskip\qquad A = $sin(3.00 \cdot ({x}^{5.00}))$

\qquad B = $cos(({x}^{6.00}) \cdot 3.00)$

имеет вид:

\begin{figure}[h]\center{\includegraphics[width=100mm]{graph.png}}\label{fig:t}\end{figure}Уравнение касательной в точке x=2.00 имеет вид:

y = -413.84x + 828.54:

\bigskip Иииииииииииииии если:

$({x}^{6.00})'$ = $1.00 \cdot A$, где:

\bigskip\qquad A = $6.00 \cdot ({x}^{(6.00 - 1.00)})$



\bigskip Вас это не шокирует?

$(({x}^{6.00}) \cdot 3.00)'$ = $(1.00 \cdot A) \cdot 3.00 + ({x}^{6.00}) \cdot 0.00$, где:

\bigskip\qquad A = $6.00 \cdot ({x}^{(6.00 - 1.00)})$



\bigskip Ничего не понял, но очень интересно:

$(cos(({x}^{6.00}) \cdot 3.00))'$ = $(-1.00 \cdot ((1.00 \cdot B) \cdot 3.00 + ({x}^{6.00}) \cdot 0.00)) \cdot A$, где:

\bigskip\qquad A = $sin(({x}^{6.00}) \cdot 3.00)$

\qquad B = $6.00 \cdot ({x}^{(6.00 - 1.00)})$



\bigskip Очередное халявное преобразование:

$({(cos(({x}^{6.00}) \cdot 3.00))}^{2.00})'$ = $((-1.00 \cdot ((1.00 \cdot C) \cdot 3.00 + ({x}^{6.00}) \cdot 0.00)) \cdot B) \cdot (2.00 \cdot ({A}^{(2.00 - 1.00)}))$, где:

\bigskip\qquad A = $cos(({x}^{6.00}) \cdot 3.00)$

\qquad B = $sin(({x}^{6.00}) \cdot 3.00)$

\qquad C = $6.00 \cdot ({x}^{(6.00 - 1.00)})$



\bigskip Совершенно очевидно, что

$({x}^{5.00})'$ = $1.00 \cdot A$, где:

\bigskip\qquad A = $5.00 \cdot ({x}^{(5.00 - 1.00)})$



\bigskip Совершенно очевидно, что

$(3.00 \cdot ({x}^{5.00}))'$ = $0.00 \cdot ({x}^{5.00}) + 3.00 \cdot (1.00 \cdot A)$, где:

\bigskip\qquad A = $5.00 \cdot ({x}^{(5.00 - 1.00)})$



\bigskip Заметим, что

$(sin(3.00 \cdot ({x}^{5.00})))'$ = $B \cdot (0.00 \cdot ({x}^{5.00}) + 3.00 \cdot (1.00 \cdot A))$, где:

\bigskip\qquad A = $5.00 \cdot ({x}^{(5.00 - 1.00)})$

\qquad B = $cos(3.00 \cdot ({x}^{5.00}))$



\bigskip Это преобразование позаимствуем из вступительных испытаний в советские ясли:

$(({(cos(({x}^{6.00}) \cdot 3.00))}^{2.00}) \cdot (sin(3.00 \cdot ({x}^{5.00}))))'$ = $(((-1.00 \cdot ((1.00 \cdot G) \cdot 3.00 + ({x}^{6.00}) \cdot 0.00)) \cdot F) \cdot (2.00 \cdot ({E}^{(2.00 - 1.00)}))) \cdot D + ({C}^{2.00}) \cdot (B \cdot (0.00 \cdot ({x}^{5.00}) + 3.00 \cdot (1.00 \cdot A)))$, где:

\bigskip\qquad A = $5.00 \cdot ({x}^{(5.00 - 1.00)})$

\qquad B = $cos(3.00 \cdot ({x}^{5.00}))$

\qquad C = $cos(({x}^{6.00}) \cdot 3.00)$

\qquad D = $sin(3.00 \cdot ({x}^{5.00}))$

\qquad E = $cos(({x}^{6.00}) \cdot 3.00)$

\qquad F = $sin(({x}^{6.00}) \cdot 3.00)$

\qquad G = $6.00 \cdot ({x}^{(6.00 - 1.00)})$



\bigskip После очевидных упрощений имеем:

$(((-1.00 \cdot G) \cdot F) \cdot (2.00 \cdot E)) \cdot D + ({C}^{2.00}) \cdot (B \cdot A)$, где:

\bigskip\qquad A = $3.00 \cdot (5.00 \cdot ({x}^{4.00}))$

\qquad B = $cos(3.00 \cdot ({x}^{5.00}))$

\qquad C = $cos(({x}^{6.00}) \cdot 3.00)$

\qquad D = $sin(3.00 \cdot ({x}^{5.00}))$

\qquad E = $cos(({x}^{6.00}) \cdot 3.00)$

\qquad F = $sin(({x}^{6.00}) \cdot 3.00)$

\qquad G = $(6.00 \cdot ({x}^{5.00})) \cdot 3.00$


\end{document}