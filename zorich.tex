\documentclass{article}

\usepackage{amssymb, amsmath, multicol}
\usepackage{graphicx}
\usepackage{float}
\usepackage{wrapfig}
\usepackage[utf8]{inputenc}
\usepackage[T1,T2A]{fontenc}
\usepackage[russian]{babel}
\usepackage{minibox}
\begin{document}
\center
Дано: $-1x$

\bigskip Ну что? Тейлора тебе дать?

\minibox[frame]{$\frac{-1}{1} \cdot {x}^{1} + \overline{\overline{o}}({x}^{3})$}

\bigskip График функции $-1x$имеет вид:

\begin{figure}[h]\center{\includegraphics[width=100mm]{graph.png}}\label{fig:t}\end{figure}

 \minibox[frame]{\centerline{Уравнение касательной в точке x=0 имеет вид:}\\
\centerline{y = -0x + 0}}

\bigskip Ну вот! 14 стадий, и матан выучен!

$(-1x)'$ = $0x + -1 \cdot 1$

\bigskip После очевидных упрощений имеем:

$-1$
\end{document}